\section{Szymon Kowalski}
\label{sec;Szymon Kowalski}

\subsection{Zdjęcie}
PETER Opis nie wymagany
\begin{figure}[htbp]
 \centering
 \includegraphics[scale= 0.7]{pictures/PETER.jpg} 
 \caption{PETER }
 \label{fig:PETER}
\end{figure}



\subsection{Wzorek}
\[cos(nt)=\frac{((cos(t)+isin(t))^n+(cos(t)-isin(t))^n)}{22  + t + t^2 + 2^t+t^3 + 3^t}\]

\subsection{Listy}
Produkcje Walaszka;
\begin{itemize}
    \item Kaptian Bomba
    \item Blok ekipa
    \item Egzorcysta
    \item Galaktyczne lektury 
    \item Laserowy gniew dzidy
    \item Generał Italia
\end{itemize}

\subsection{Tekst}
\underline{\textbf{\textit{- Zło to zło, Stregoborze-}}}
\emph{rzekł poważnie wiedźmin wstając.}
Mniejsze, większe, średnie, wszystko jedno,
proporcje są umowne a granice zatarte. 
Nie jestem świątobliwym pustelnikiem, 
nie samo dobro czyniłem w życiu. 
Ale jeżeli mam wybierać pomiędzy jednym złem a drugim, 
to wolę nie wybierać wcale.
\\\textbf{\textit{Andrzej Sapkowski, Ostatnie życzenie}}
\newline
\underline{\textbf{\textit– Kompania mi się trafiła – }}
podjął Geralt, kręcąc głową. 
\emph{– Towarzysze broni! Drużyna bohaterów!}
Nic, tylko ręce załamać. Wierszokleta z lutnią.
Dzikie i pyskate pół driady, pół baby. 
Wampir, któremu idzie na pięćdziesiąty krzyżyk. 
I cholerny Nilfgaardczyk, 
który upiera się, że nie jest Nilfgaardczykiem.
\\\textbf{\textit{Andrzej Sapkowski, Chrzest Ognia}}


\subsection{Tabela}
\label{tab:Piwo}
\begin{table}[htbp]
\centering
\begin{tabular}{lllll}
\hline

1 & 1   & 2    & 5    & e     \\ \hline
2 & 1   & 4    & 25   & 784   \\ \hline
3 & 5   & 1421 & 2421 & 21    \\ \hline
4 & 12  & 37   & 6    & 9     \\ \hline
5 & 21  & 420  &  --  &  --   \\ \hline
  &     &      &      &       \\ 
\end{tabular}
\end{table}

\subsection{Odwołania}
 Peter (\ref{fig:PETER}) nie wie że tu jest i lepiej żeby tak zostało D: