\section {Martyna Gaj}
\label{sec:Martyna Gaj}

\subsection{Zdjęcie}
Dodaję zdjęcie fajnego kota
(see Figure)~\ref{fig:Chonk}
\begin{figure}[htbp]
 \centering
 \includegraphics[scale= 0.5]{pictures/Chonk.jpg} 
 \caption{Oh God he's chonky}
 \label{fig:Chonk}
\end{figure}

\subsection{Tabela}
\label{tab:marg}
\begin{table}[htbp]
\centering
\begin{tabular}{lllll}
\hline

1 & 34   & 87654 & 23    & 85  \\ \hline
2 & 5    & 987   & 654   & 784 \\
3 & 4354 & 243   & 45    & 32  \\
4 & 234  & 765   & 2190  & 329 \\
5 & 1    & 78    & 23467 & 6   \\
  &      &       &       &     \\ \hline
\end{tabular}
\end{table}

\subsection{Wzory}
Bardzo lubię matematykę, w której są cyferki
\[x^2=36\]
albo przynajmniej gdzie coś się liczy
$ x^3 + 4x^2 + 7x -14 = 53 $

\subsection{Listy}
\textbf{Moje ostatnie wyszukiwania:}
\begin{itemize}
    \item[!] Jakie studia wybrać?
    \item[;(] Co ja robię ze swoim życiem?
    \item Przepis na zupę-krem czosnkową
\end{itemize}

\textbf{Moje koty:}
\begin{enumerate}
    \item Garfield
    \item Tofik
    \item Bożena
    \item Albin
\end{enumerate}

\subsection{Tekst}
\emph{W kociołkach bigos grzano}; w słowach wydać trudno
Bigosu smak przedziwny, kolor i woń cudną;
\textbf{Słów tylko brzęk usłyszy i rymów porządek,}
Ale treści ich miejski nie pojmie żołądek.
Aby cenić litewskie pieśni i potrawy,
Trzeba mieć zdrowie, na wsi żyć, wracać z obławy.
Przecież i bez tych przypraw potrawą nie lada
\textbf{\textit{Jest bigos, bo się z jarzyn dobrych sztucznie składa.}}
Bierze się doń siekana, kwaszona kapusta,
Która, wedle przysłowia, sama idzie w usta;
Zamknięta w kotle, łonem wilgotnym okrywa
Wyszukanego cząstki najlepsze mięsiwa;
I praży się, aż ogień wszystkie z niej wyciśnie
Soki żywne, aż z brzegów naczynia war pryśnie
I powietrze dokoła zionie aromatem.
\underline{\textbf{\textit{Bigos już gotów}}}


\subsection{Odwołania}
Pamiętajmy o: kocie, którego znajdziesz tutaj \ref{fig:Chonk} oraz o tabeli, którą znajdziesz tu \ref{tab:marg}

